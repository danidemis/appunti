\documentclass[11pt]{article}
\usepackage[utf8]{inputenc}
\usepackage[italian]{babel}
\usepackage{amssymb}
\usepackage{verbatim}
\usepackage{amsfonts}
\usepackage{amsmath}
\usepackage{amsthm}
\usepackage{xcolor}
\usepackage{graphicx}

\begin{comment}
\title{Linguaggi e Computabilità}
\author{Daniele De Micheli}
\date{2019}

\renewcommand*\contentsname{\textit{Indice}}
\end{comment}

%presettaggio per teoremi e assiomi/definizioni%

\newtheorem*{problemaDecisione}{Problema di Decisione}
\newtheorem*{problemaMembership}{Problema di Membership}

%fine%

\begin{document}

%\maketitle
%\tableofcontents

\subsection{Linguaggio} possiamo definire un \textit{linguaggio} $L \medspace su \medspace E$ un sottoinsieme di $E^{\star}$ tale che $L \subseteq E^{\star}$. Per esempio, preso $E = \{a,b,c\}$, un linguaggio L potrebbe essere $L_{1}=\{aa,cbc\}$. Un linguaggio può essere finito (vedi $L_ {1}$), oppure infiniti (es. $L_{2} = \{ w \in E^{\star} \thickspace | \thickspace w \medspace contiene \medspace lo \medspace stesso \medspace numero \medspace di \medspace a \medspace e \medspace c\}$).
\\ \\
Preso un linguaggio $L \subseteq E^{\star}$, possiamo affermare che:
\begin{itemize}
\item[1.] $ \emptyset \subseteq L$;
\item[2.] $ \varepsilon \subseteq L $;
\item[3.] $ E^{\star} \subseteq L$;
\end{itemize}
sono tutti linguaggi. La principale caratteristica di un linguaggio è che esso deve essere riconosciuto e interpretato da una macchina (o automa) ed essa deve anche essere in grado di generarlo tramite una \textbf{\textit{grammatica}}.
\begin{problemaDecisione}
Il problema di decisione si presenta nel momento in cui, dato un quesito, le possibili risposte sono sempre e sole "sì" o "no".
\end{problemaDecisione}
\begin{problemaMembership}
Il problema di Memebership è legato al concetto di stringa (come input), di linguaggio e di appartenenza ad un determinato linguaggio. Data una stringa w in input, una determinata macchina deve essere in grado di dire se essa appartiene ad un linguaggio oppure no.
\end{problemaMembership}

\textbf{\textit{DEFINIZIONI}}
Una \textbf{forma sentenziale} è una stringa di simboli terminali e non terminali: $\gamma \in (V \cup T)^{\star}$
\subsection{Grammatica context-free -CFG-}
Una grammatica context free è una grammatica che non prevede l'incrocio dei simboli terminali per cui è necessario utilizzare delle regole differenti. Un esempio di linguaggio context free è il seguente:
\\ \\
\paragraph{Stringhe palindrome}: le stringhe palindrome sono un esempio semplice di linguaggio che utilizza una grammatica context-free.
Abbiamo il l'alfabeto $E = \{0,1\}$ e il linguaggio costruito su esso $L_{pal} \subseteq E^{\star}$. Da questo alfabeto e con questo linguaggio possiamo costruire una stringa w palondroma come $$w = \{0110\}$$
Essa può essere definita per induzione come segue:
\begin{enumerate}
\item Caso base: $\varepsilon , 0, 1 \in L_{pal}$
\item Caso induttivo: se $w \in L_{pal}$, allora $0w0, 1w1, \varepsilon \in L_{pal}$
\end{enumerate}
\subsection{Grammatica NON context-free}
Il linguaggio di esempio (di tipo 2): $L = \{w \in \{a,b,c\}^{\star} | w = a^nb^nc^n, n\geq 1\}$
è generato dalla seguente grammatica (NON context-free): $$ G=(\{S,X,B,C\}, \{a,b,c\}, P, S)$$
e dove le regole di produzione sono:
\begin{enumerate}
	\item $S \rightarrow aSBC$
	\item $S \rightarrow aBC$
	\item $CB \rightarrow XB$
	\item $XB \rightarrow XC$
	\item $XC \rightarrow BC$
	\item $aB \rightarrow ab$
	\item $bB \rightarrow bb$
	\item $bC \rightarrow bc$
	\item $cC \rightarrow cc$
\end{enumerate}
Le grammatuche 3,4,5 possono essere "collassate" in $CB \rightarrow BC$
Si può dimostrare , usando il Pumping Lemma per i CFL, che non è context-free.
\\ \\
Esempio di Derivazione:
\\ \\
Deriviamo la stringa abc (corrispondente a n = 1), indicando anche ad ogni passo la regola usata.
$$ S (2)\rightarrow aBC (6)\rightarrow abC (8)\rightarrow abc$$
Deriviamo la stringa aabbcc (corrispondente a n = 1), indicando anche ad ogni passo la regola usata.

\begin{equation*}\label{stigeiz}\begin{split}
S (1)\rightarrow aSBC (2)\rightarrow aaBCBC (3)\rightarrow aaBXBC (4) \rightarrow aaBXCC (5) \rightarrow \\&\\ \rightarrow aaBBCC (6)\rightarrow aabBCC (7)\rightarrow aabbCC (8)\rightarrow aabbcC (9)\rightarrow aabbcc
\end{split}
\end{equation*}

In generale, per derivare $a^nb^nc^n$, per n $<$ 1:

\begin{equation*}
\begin{split}
S(n-1 \medspace volte \rightarrow (1)) a^{n-1}S(BC)^{n-1}\rightarrow (2) a^n(BC)^n(n(n-1)/2 \medspace volte \medspace la \\&\\ sequenza \rightarrow (3), \rightarrow (4), \rightarrow (5))a^nB^nC^n....slide
\end{split}
\end{equation*}

Esercizio: creo una CFG su $L = \{a^{n+m}xc^nyd^m, \medspace con n,m \geq 0\}$:





\end{document}
